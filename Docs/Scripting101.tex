\documentclass[12pt,a4paper]{article}

\usepackage{listings}
\usepackage{xcolor}
\usepackage[colorlinks=true]{hyperref}
\usepackage{makeidx}

\definecolor{color_keyword}{rgb}{0,0.6,0}
\definecolor{color_comment}{rgb}{0.5,0.5,0.5}
\definecolor{color_string}{rgb}{0.58,0,0.82}
\definecolor{color_number}{rgb}{0.58,0,0.82}
\definecolor{color_background}{rgb}{0.95,0.95,0.95}

\lstdefinestyle{mystyle}{
    backgroundcolor=\color{color_background},   
    commentstyle=\color{color_comment},
    keywordstyle=\color{color_keyword},
    numberstyle=\tiny\color{color_number},
    stringstyle=\color{color_string},
    basicstyle=\ttfamily\footnotesize,
    breakatwhitespace=false,
    breaklines=true,
    captionpos=b,
    keepspaces=true,
    numbers=left,
    numbersep=5pt,
    showspaces=false,
    showstringspaces=false,
    showtabs=false,                  
    tabsize=3
}
\lstset{style=mystyle}

\title{Engine Scripting with Lua\\(WIP)}
\author{\texttt{@lionkor}}

\begin{document}

\begin{titlepage}
\clearpage\maketitle
\thispagestyle{empty}
\end{titlepage}

\pagenumbering{roman}

\tableofcontents
\pagebreak

\begin{abstract}
This engine lets you script behavior in \href{https://www.lua.org/about.html}{Lua}. 
In order for this to be useful, the engine provides a number of functions and globals, which are documented in this PDF. Further, it will show how to use Lua to script some basic behavior by providing some examples.

\end{abstract}

\pagebreak

\pagenumbering{arabic}
\setcounter{page}{1}

\section{Setup}

For scripts to be read by the engine, an \texttt{Entity} needs to have a  \texttt{ScriptableComponent}. The constructor of the latter accepts a scriptfile name, like \texttt{example.lua}. For the file to be loaded it needs to be in the \texttt{Data/} directory and listed in the \texttt{Data/res.list} of your project.

An example program follows that will be used for the rest of this pdf.

\begin{lstlisting}[language=C++,title=example.cpp]
#include "Engine.h"
int main() {
	// Create an application
	Application app("Scripting101 Program", { 800, 600 });
	// Create an entity
	WeakPtr<Entity> entity_weak = app.add_entity();
	// Lock the Ptr for temporary access
	auto entity = entity_weak.lock();
	// Add ScriptableComponent
	entity->add_component<ScriptableComponent>("example.lua");
	// Run the application
	return app.run();
}
\end{lstlisting}

Additionally, the files \texttt{Data/example.lua} and \texttt{Data/res.list} exist.

\begin{lstlisting}[language={[5.0]Lua},title=Data/example.lua]
Engine.log_info("Hello, World!")
\end{lstlisting}

\begin{lstlisting}[title=Data/res.list]
example.lua
\end{lstlisting}

With these files in place and the \texttt{example.cpp} compiled, we can now write code in \texttt{Data/example.lua}.

\pagebreak
\section{Exposed Functions}

\subsection{Entity API}

The \texttt{Entity} namespace provides access to the entity that this component belongs to. 

\subsubsection{Entity.position()}
\begin{itemize}
	\item[]{\bf Description}
		\\ Returns the position of the entity.
	\item[]{\bf Arguments}
		\\ None
	\item[]{\bf Returns}
		\begin{enumerate}
			\item{\texttt{x - number}} 
				\\ The x-component of the position.
			\item{\texttt{y - number}} 
				\\ The y-component of the position.
		\end{enumerate}
\end{itemize}

\subsubsection{Entity.rotation()}
\begin{itemize}
	\item[]{\bf Description}
		\\ Returns the rotation of the entity.
	\item[]{\bf Arguments}
		\\ None
	\item[]{\bf Returns}
		\begin{enumerate}
			\item{\texttt{r - number}} 
				\\ The rotation of the entity.
		\end{enumerate}
\end{itemize}

\subsubsection{Entity.move\_by(dx, dy)}\label{EntityMoveBy}
\begin{itemize}
	\item[]{\bf Description}
		\\ Moves the entity by a specific amount of units. Use \texttt{Entity.set\_position} (see \ref{EntitySetPosition}) to set the position directly.
	\item[]{\bf Arguments}
		\begin{enumerate}
			\item{\texttt{dx - number}} 
				\\ Change (delta) in x-direction.
			\item{\texttt{dy - number}} 
				\\ Change (delta) in y-direction.				
		\end{enumerate}
	\item[]{\bf Returns}
		\\ Nothing
\end{itemize}

\subsubsection{Entity.set\_position(x, y)}\label{EntitySetPosition}
\begin{itemize}
	\item[]{\bf Description}
		\\ Moves the entity to a specific position. Use \texttt{Entity.move\_by} (see \ref{EntityMoveBy}) to move the entity by an amount.
	\item[]{\bf Arguments}
		\begin{enumerate}
			\item{\texttt{x - number}} 
				\\ New x-coordinate.
			\item{\texttt{y - number}} 
				\\ New y-coordinate.
		\end{enumerate}
	\item[]{\bf Returns}
		\\ Nothing
\end{itemize}

\subsection{Engine API}

The \texttt{Engine} namespace provides general engine functionality, mostly used for debugging and core engine functionalities.


\subsubsection{Engine.log\_info(message)}
\begin{itemize}
	\item[]{\bf Description}
		\\ Prints an info message to the debug output.
	\item[]{\bf Arguments}
	\begin{enumerate}
		\item{\texttt{message - string}} 
			\\ The message to be printed.
	\end{enumerate}
	\item[]{\bf Returns}
		\\ Nothing
\end{itemize}

\subsubsection{Engine.log\_warning(message)}
\begin{itemize}
	\item[]{\bf Description}
		\\ Prints a yellow warning message to the debug output.
	\item[]{\bf Arguments}
	\begin{enumerate}
		\item{\texttt{message - string}} 
			\\ The message to be printed.
	\end{enumerate}
	\item[]{\bf Returns}
		\\ Nothing
\end{itemize}

\subsubsection{Engine.log\_error(message)}
\begin{itemize}
	\item[]{\bf Description}
		\\ Prints a red error message to the debug output.
	\item[]{\bf Arguments}
	\begin{enumerate}
		\item{\texttt{message - string}} 
			\\ The message to be printed.
	\end{enumerate}
	\item[]{\bf Returns}
		\\ Nothing
\end{itemize}

\pagebreak
\section{Constants}

The engine exposes several constant values and tables to all scripts. These are read-only.
\subsection{Tables}

\subsubsection{MouseButton}

\begin{itemize}
	\item[]{\bf Description}
		\\ A table which holds the integer values used in identifying mouse buttons in mouse-event related callbacks. The actual values are implementation defined and may change.
	\item[]{\bf Entries}
\begin{itemize}
	\item{\texttt{MouseButton.LMB}} - Left mouse button integer equivalent
	\item{\texttt{MouseButton.RMB}} - Right mouse button integer equivalent
	\item{\texttt{MouseButton.MMB}} - Middle mouse button integer equivalent
	\item{\texttt{MouseButton.XB1}} - Extra mouse button 1 integer equivalent
	\item{\texttt{MouseButton.XB2}} - Extra mouse button 2 integer equivalent
	
\end{itemize}
\end{itemize}

\subsection{Values}

Please note that values with "unfriendly" names such as \texttt{g\_parent} are only to be used internally, but are documented here to provide a single and complete reference.

\subsubsection{g\_scriptfile\_name}
\begin{itemize}
	\item[]{\bf Description}
		\\ The full name of the current script file. Used automatically by the engine in calls to \texttt{Engine.log\_*} and similar function families.
\end{itemize}

\subsubsection{g\_parent}
\begin{itemize}
	\item[]{\bf Description}
		\\ The address of the parent \texttt{Entity}, as \texttt{std::uintptr\_t}. Used internally in the implementation of parent-modifying functions. May be used to compare entities in a really crude way.
\end{itemize}

\pagebreak
\section{Special Lua Functions}

The engine calls specific lua functions (if they exist), at specific points in time or when events occur. The following is a listing of all of those special functions.
If they do not exist, a warning is printed into the debug output once and any further attempts at calling the function will not occur.

\subsection{Periodically Called Functions}

These functions are called periodically. It is advised not to put any heavy calculations in any of these, unless absolutely unavoidable. Any "power-hungry" code in these functions will cause a slowdown.

\subsubsection{update()}
\begin{itemize}
	\item[]{\bf Description}
		\\ Called every frame from the moment the \texttt{ScriptableComponent} is created until it or the \texttt{Entity} is destroyed.
	\item[]{\bf Arguments}
		\\ None
	\item[]{\bf Returns}
		\\ Nothing
	\item[]{\bf Example}
\begin{lstlisting}[language={[5.0]Lua}]
function update()
	-- do things here
end
\end{lstlisting}
\end{itemize}


\subsection{Event Callbacks}

These functions are called when a specific event occurs, for example a mouse click.

\subsubsection{on\_mouse\_down(mouse\_button, x, y)}
\begin{itemize}
	\item[]{\bf Description}
		\\ Called when the user presses any mouse button.
	\item[]{\bf Arguments}
		\begin{enumerate}
			\item{\texttt{mouse\_button - integer}} 
				\\ The mouse button that has been pressed. The buttons are represented as integers, but the engine provides global \texttt{MouseButton} table to compare against:
				\begin{itemize}
					\item[-]\texttt{MouseButton.LMB} - The left mouse button
					\item[-]\texttt{MouseButton.RMB} - The right mouse button
					\item[-]\texttt{MouseButton.MMB} - The middle mouse button
					\item[-]\texttt{MouseButton.XB1} - The extra button 1 (only exists on some mice)
					\item[-]\texttt{MouseButton.XB2} - The extra button 1 (only exists on some mice)
				\end{itemize}
			\item{\texttt{x - number}}
				\\ The x-position of the mouse in the world.
			\item{\texttt{y - number}}
				\\ The y-position of the mouse in the world.
		\end{enumerate}
	\item[]{\bf Returns}
		\\ Nothing
	\item[]{\bf Example}
	\\ This example prints "\texttt{lmb left mouse pressed!}" in the debug output whenever the user presses the left mouse button.
\begin{lstlisting}[language={[5.0]Lua}]
function on_mouse_down(mb, x, y)
    if mb == MouseButton.LMB then
        Engine.log_info("left mouse pressed!")
    end
end
\end{lstlisting}
\end{itemize}

\end{document}